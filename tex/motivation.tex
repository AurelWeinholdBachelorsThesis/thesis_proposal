\section{Motivation}

\todo{First two sentences start with \enquote{As something gets bigger}. Change
    that!}

% IIoT
As industrial factories become larger and more connected, the demand for a
decentralized distributed system consisting of sensors, controllers and
compute-nodes grows larger over time.
%This describes the Industrial Internet of Things (IIoT), an umbrella term for
%networks consisting of many small devices, such as sensors or robots, possibly
%connected, coordinated and controlled by a cloud.
The Industrial Internet of Things is defined by
\citeauthor{boyes_industrial_2018} to be a \enquote{system of networked smart
    objects, cyber physical assets, [such as sensors or robots,] [$\dots$] and
    optional cloud or edge computing platforms, which enable real time,
    intelligent, and autonomous access, collection, analysis, communications,
    and exchange of process, product and/or service information, within the
    industrial environment [$\dots$].}%so as to optimize overall production value.}


% edge
\todo{First two sentences start with \enquote{As something gets bigger}. Change
    that!}
As these individual IIoT networks grow, more resources are necessary for the
networking, as well as the controlling component. To help reduce this, edge
computing, a new paradigm in distributed systems, has introduced the idea to
(pre-)compute \enquote{on the edge} of the cloud, which might significantly
reduce stress on the network and on the compute-resources of the cloud itself.
\todo{Statistics on IIoT network growth}
\todo{Statistics on network congestion in larger networks - why edge?}
This edge computing is always subject to improvement benefiting the entire network.

% real time
Some control also exhibit real time behavior. These especially benefit from
faster response times and sorting based on priority, enabling them to meet their
hard deadlines.

% eBPF
Extended Berkeley Packet Filter (eBPF) is a flexible virtual machine in the
Linux kernel programmable during runtime from user space. It allows for simple
programs to be safely executed in the kernel space at various hook points. These
simple programs may be written in a subset of C and then compiled into eBPF
instructions, but similar to assembly, it is entirely possible to write the
programs in eBPF instructions yourself. When loaded into the kernel, they are
tested by the in-kernel verifier. This verifier checks the program against a set
of restrictions including a limitation on the number of (eBPF) instructions and
that there are not loops in the
program\footnote{\url{https://www.kernel.org/doc/html/latest/bpf/verifier.html},
accessed \formatdate{7}{6}{2022}} to
ensure the safety of the kernel. Some of these restrictions however can be
circumvented without diminishing the safety of the kernel.
\todo{Rewrite part about loops.}



%But as these networks grow, traffic and the need for computing
%power is increasing as well. This is where edge computing comes into play,
%trying to reduce computational needs in the cloud and traffic into and out of
%the cloud, by precomputing and presorting network traffic right on the edge of
%the cloud. As this is a time sensitive task it might be worth exploring
%possibilities to speed it up.
