\section{Motivation}

\todo{First two sentences start with \enquote{As something gets bigger}. Change
    that!}

% IIoT
As industrial factories become larger and more connected, the demand for a
decentralized distributed system consisting of sensors, controllers and
compute-nodes grows larger over time.

The Industrial Internet of Things, as defined by
\citeauthor{boyes_industrial_2018}, is the aggregation of networked sensors and
actuators with optional cloud or edge computing platforms which enable real
time, intelligent, and autonomous access, collection, analysis, and
communications, of process, product and service information, within the
industrial environment~\cite{boyes_industrial_2018}, where industrial means
manufacturing and excludes mining, construction and
energy~\cite{noauthor_industry_nodate}.

% real-time
\todo{weird first sentence.}
Some of these \enquote{things} also exhibit real-time behavior, to ensure safety
and Quality of Service (QoS). Real-time operation is defined by DIN 44300 to be
the operation of a computer system with programs that are at any time ready to
process data in a way that the computational results are available within a
given period of time. This is necessary as these actors actually interact with
the real world, having to reliably interact with others and possible even
humans. To mitigate risk for human workers and to ensure QoS in the product it
is strictly necessary to be dependable, as a violation might cost millions or
potentially the live of humans.
%\todo{Add actual source.}
%\todo{predictability}
%\todo{more}

%These \enquote{things} work depending on hard deadlines might not be able to
%rely on the unknown network connecting it with the cloud.

% edge
As latencies in unknown networks are unpredictable a shift away from the cloud
towards the IIoT networks has been made. This edge computing, a new paradigm in
distributed systems, has introduced the idea to (pre-)compute \enquote{on the
edge} of the cloud~\cite{shi_edge_2016}, which significantly reduces response
times by avoiding these network latencies, all the while also reducing stress on
networks and the computing resources of the cloud.

%\todo{Is global market actually IIoT? Is this relevant?}
%According to \citeauthor{placek_industrial_nodate} the global market for IIoT is
%expected to grow almost five-fold up to 2028~\cite{placek_industrial_nodate} and
%with it there will be an increase in demand for networking and controlling
%resources. To combat this fast increase, edge computing, a new paradigm in
%distributed systems, has introduced the idea to (pre-)compute \enquote{on the
%edge} of the cloud~\cite{shi_edge_2016}, which significantly reduces stress on
%the network and on the compute-resources of the cloud itself, while also
%allowing for faster response times by avoiding congested
%networks~\cite{shi_edge_2016}.


%But as these networks grow, traffic and the need for computing
%power is increasing as well. This is where edge computing comes into play,
%trying to reduce computational needs in the cloud and traffic into and out of
%the cloud, by precomputing and presorting network traffic right on the edge of
%the cloud. As this is a time sensitive task it might be worth exploring
%possibilities to speed it up.
