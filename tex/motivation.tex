\section{Motivation}

\todo{First two sentences start with \enquote{As something gets bigger}. Change
    that!}

% IIoT
As industrial factories become larger and more connected, the demand for a
decentralized distributed system consisting of sensors, controllers and
compute-nodes grows larger over time.

The Industrial Internet of Things, as defined by
\citeauthor{boyes_industrial_2018}, is the aggregation of networked sensors and
actuators with optional cloud or edge computing platforms which enable real
time, intelligent, and autonomous access, collection, analysis, and
communications, of process, product and service information, within the
industrial environment~\cite{boyes_industrial_2018}, where industrial means
manufacturing and excludes mining, construction and
energy~\cite{noauthor_industry_nodate}.

% edge
\todo{Is global market actually IIoT? Is this relevant?}
According to \citeauthor{placek_industrial_nodate} the global market for IIoT is
expected to grow almost five-fold up to 2028~\cite{placek_industrial_nodate}
and with it there will be an increase in demand for networking and controlling
resources. To combat this fast increase, edge computing, a new paradigm in
distributed systems, has introduced the idea to (pre-)compute \enquote{on the
edge} of the cloud~\cite{shi_edge_2016}, which significantly reduces stress on
the network and on the compute-resources of the cloud itself, while also
allowing for faster response times by avoiding congested
networks~\cite{shi_edge_2016}.

% real time
Some control also exhibit real time behavior. These especially benefit from
faster response times and sorting based on priority, enabling them to meet their
hard deadlines.

% eBPF
Extended Berkeley Packet Filter (eBPF) is a flexible virtual machine in the
Linux kernel programmable during runtime from user space. It allows for simple
programs to be safely executed in the kernel space at various hook points. These
simple programs may be written in a subset of C and then compiled into eBPF
instructions, but similar to assembly, it is entirely possible to write the
programs in eBPF instructions yourself. When loaded into the kernel, they are
tested by the in-kernel verifier. This verifier checks the program against a set
of restrictions including a limitation on the number of (eBPF) instructions and
that there are not loops in the
program\footnote{\url{https://www.kernel.org/doc/html/latest/bpf/verifier.html},
accessed \formatdate{7}{6}{2022}} to
ensure the safety of the kernel. Some of these restrictions however can be
circumvented without diminishing the safety of the kernel.
\todo{Rewrite part about loops.}



%But as these networks grow, traffic and the need for computing
%power is increasing as well. This is where edge computing comes into play,
%trying to reduce computational needs in the cloud and traffic into and out of
%the cloud, by precomputing and presorting network traffic right on the edge of
%the cloud. As this is a time sensitive task it might be worth exploring
%possibilities to speed it up.
